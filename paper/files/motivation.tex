Urban indoor environments exhibit pattern that can be exploited for exploration.
For example, hallways are often elongated rectangles that lead to other rooms
while rooms are often close to a square and a dead end.
Such patterns can be exploited for efficient exploration either for map building
or source seeking.
In this work we evaluate the advantage of learning such patterns from indoor
datasets and aiding exploration algorithms with the learned priors.
In particular, we use semantic place detection from first person view and use it
predict the room shape and the possible entropy gain from exploring such a room.
